\documentclass[10pt]{article}
\begin{document}

\title{ISS - Uebungsblatt 3}
\author{Pia Kullik (BSc Informatik, Matr.nr. 2051889) \and Robert Hemstedt (Nebenfach, Matr.nr. 2536252)}
\maketitle

\section{Aufgabe 4 - Raeumliche Transformationen}

Die Transformation gelingt durch Drehung des gegeben Vektors um 35 Grad und anschliessende Translation um $(1, 2.5)$. Bestimme die entsprechenden Matrizen und multipliziere sie zur Transformationsmatrix $K_{12}$:


\[
D_{35} =
\left( {\begin{array}{ccc}
cos(35) & -sin(35) & 0\\
sin(35 & cos(35) & 0\\
0 & 0 & 1
\end{array} } \right) 
//Drehmatrix
\]

\[
T =
\left( {\begin{array}{ccc}
1 & 0 & 1\\
0 & 1 & 2.5 \\
0 & 0 & 1
\end{array} } \right) 
//Translationsmatrix
\]

\[
\Rightarrow K_{12} = T \ast D_{35} =
\left( {\begin{array}{ccc}
cos(35) & -sin(35) & 1\\
sin(35) & cos(35) & 2.5\\
0 & 0 & 1
\end{array} } \right) 
\]


Die homogenen $K_{2}$-Koordinaten eines 2D-Punktes $\vec{v} = (2, 1.5)$ werden berechnet, indem man seine homogenen Koordinaten in $K_{1}$ mit der Transformationsmatrix $K_{12}$ multipliziert:

\[
\left( {\begin{array}{ccc}
cos(35) & -sin(35) & 1\\
sin(35) & cos(35) & 2.5\\
0 & 0 & 1
\end{array} } \right) 
\ast
\left( {\begin{array}{ccc}
2\\
1.5\\
1
\end{array} } \right) 
 = 
\left( {\begin{array}{ccc}
2cos(35) - 1.5sin(35) + 1\\
2sin(35) + 1.5cos(35) + 2.5\\
1
\end{array} } \right) 
\approx
\left( {\begin{array}{ccc}
1.778\\
4.876\\
1
\end{array} } \right)
\]





\end{document}